\section{Post quantum cryptography}
	\subsection{Post quantum computing}	
	
		\begin{frame}
			\frametitle{Quantum computing}
				 What is quantum computing?
				\begin{itemize}
					\item Computer based on principle of quantum mechanics
					\item Quantum computer contains qbits instead of bit
					\begin{itemize}
						\item Every qbit can be in state $0$ or $1$ but can also be in every \textit{sperposition} in between
						\item The state is destroyed by reading it (see Schrödingers cat)
					\end{itemize}
					\item They are able to solve computational strong problems efficient
				\end{itemize}
		\end{frame}
		
		\begin{frame}
			\frametitle{Quantum computing}
				How real is it?
				\begin{itemize}
					\item 2001: IBM Almaden Research Center realized a system with 7 Qubits 
					\begin{itemize}
						\item Factored 15 into it's prime factors 3 und 5
					\end{itemize}
					\item ...
					\item 2013: D-Wave Systems sells first quantum computes to Google and NASA
				\end{itemize}
				\vspace{10pt}
				\begin{block}{}
					\begin{center}
						\textit{
							We can say that quantum computing is a huge game changer on the filed of computation. \\
							We can't say how real they are.}
					\end{center}
				\end{block}
		\end{frame}
		
		\begin{frame}
			\frametitle{Cryptography pre- and post quantum}
			What is changing with focus on cryptography?
			\begin{itemize}
				\item Asymmetric state of the art security is no longer secure
				\begin{itemize}
					\item RSA $\rightarrow$ prime factorization
					\item ECC $\rightarrow$ discrete logarithm problem 
				\end{itemize}
				\item Symmetric algorithms are still secure
			\end{itemize}
			
			\begin{block}{}
				\begin{center}
					\textit{
					We need other asymmetric cryptographic schemes than the established ones.}
				\end{center}
			\end{block}
		\end{frame}
	
		\begin{frame}
			\frametitle{Post quantum cryptography}
			\begin{itemize}
				\item Created by Daniel J. Bernstein
				\item Algorithm loosing \textit{almost} no security executed on a quantum computer
				\item Completely different mathematical base than established ones
			\end{itemize}
		\end{frame}
		
	\subsection{Shors algorithm}
			\begin{frame}
				\frametitle{Shors algorithm}
				\begin{itemize}
					\item Developed 1994 by Peter Shor
					\item Solves prime factorization and discrete logarithm problem efficiently
					\item Monte Carlo Algorithm
					\item Classical part
					\begin{itemize}
						\item Mainly calculating gcd
					\end{itemize}
					\item Quantum part
					\begin{itemize}
						\item Mainly quantum Fouriertransformation
					\end{itemize}
				\end{itemize}
			\end{frame}
			
			\begin{frame}[fragile]
				\frametitle{Classical part}
					For n as composed number: 
										\footnotesize{
						\begin{lstlisting}
start:
 select an integer $1 < x < n$
 if gdc(x,n) is 1 // Euclidian algorithm
  return 1
 else
  r = compute_order(x) // quantum part
  if r is odd or x^(r/2) is equivalent -1(mod n)
   goto start 
  else
   return gcd(x^(r/2) - 1, n)
						\end{lstlisting}
					}
\end{frame}


			\begin{frame}[fragile]
				\frametitle{Quantum part (sketch)}
					For n as input from the classical part: 
					\footnotesize{
					\begin{lstlisting}
 start:
 Determine q as power of 2 with n^2 <= q <= 2n^2
 Init the input register with superposition of a mod q
 Init the output  register with x^a(mod n)
 Perform quantum Fouriertransformation on input register
 r = meassurement of output register
 if r != order(x)
  goto start
 else
  return r
					\end{lstlisting} }
Note: The input quantum reg has all possible states of a mod.
\end{frame}
						
			\begin{frame}
				\frametitle{Complexity considerations}
				\begin{itemize}
					\item $O((log n)^3)$ Instructions
					\item Complexity class of BQP
					\item 
					
				\end{itemize}
			\end{frame}
			
	\subsection{Canditates for post quantum cryptography}
	
			\begin{frame}
				\frametitle{Lattice-based cryptography}
				\begin{itemize}
					\item Came um in 1996
					\item Lattice over a n-dimensional finite Euclidian field
					\begin{itemize}
						\item Strong peridicity required
					\end{itemize}
					\item Set of vectors setting up the base
					\item Unique representation
					\item Cryptographic problem: Finding closest vector to an lattice point
					%Algorithm works on a lattice over a n-dimensional finite Euclidian field L with an strong periodicity property. A set of vectors sets up the basis of L in the way that every element is uniquely represented. The cryptographic problem is to find the closest vector to an given lattice point e.g. by adding an error vector
					\item 
				\end{itemize}
			\end{frame}

			\begin{frame}
				\frametitle{Multivariate cryptography}
				\begin{itemize}
					\item First mentione 1988
					\item Multivariate polynomials over a finite field F
					\item Defined over both a ground and an extension field
					\item Promising for digital signatures
					\item Private key consist of two affine transformations having an group endomorphism
					\item Public key is the concatination of them
				\end{itemize}
				
				% Multivariate cryptography is based on a multivariate polynomials over a finite field F which are defined over both a ground and an extension field. In case of Solving systems they are NP-complete and due to this fact a candidate for post quantum cryptography. They are topic of studies for a long time too and promising especially for signature schemes[1][3].
				
			\end{frame}
			
			\begin{frame}
				\frametitle{Hash-based cryptography}
				\begin{itemize}
					\item Created by Lamport and Merkle in ~1979
					\item Only usable for digital signatures
					
					\item Hash-based cryptographic algorithms
					\begin{itemize}
						\item PQ resistance required
					\end{itemize}

					\item Limited count of signatures for one key

					
				\end{itemize}
				%Hash-based cryptography Hash-based algorithms such	as Lamport-[4] and the Merkle[5] ignature scheme is based on strong hash functions but has the downside that only a limited count signatures can be created per key. The algorithm reduces the one time signature to an hash value unsing a hash	function[1].
				
			\end{frame}
			
			\begin{frame}
				\frametitle{Code-based cryptography}
				\begin{itemize}
					\item Founded by Robert McEliece in 1978
					\item Based on error correcting codes
					\begin{itemize}
						\item Goppa Codes
						\item Irreducibility
					\end{itemize}
					\item Good for encryption
					\item Difficult for signing
				\end{itemize}
			\end{frame}
	
	